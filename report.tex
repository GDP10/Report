\documentclass[a4paper, 12pt]{article}
\usepackage[backend=biber]{biblatex}
\usepackage{graphicx}

\addbibresource{report.bib}
\nocite{*}

\begin{document}

\begin{titlepage}

	\center

	{\large ELECTRONICS AND COMPUTER SCIENCE}\\[0.2cm]
	{\large FACULTY OF PHYSICAL SCIENCES AND ENGINEERING}\\[0.2cm]
	{\large UNIVERSITY OF SOUTHAMPTON}\\[3cm]

	{\Large Peter Prince}\\[0.2cm]
	{\Large James Robinson}\\[0.2cm]
	{\Large Charles Sherman}\\[0.2cm]
	{\Large Andrew Sullivan}\\[1cm]
	{\Large \today}\\[3cm]

	{\LARGE Aeronautical Event Service}\\[3cm]

	{\large Project Supervisor: Dr Robert Walters}\\[0.2cm]
	{\large Second Examiner: -}\\[2.5cm]

	{\large A project report submitted for the award of}\\[0.2cm]
	{\Large MEng Computer Science}

\end{titlepage}

\begin{abstract}

\end{abstract}

\newpage

\section*{Acknowledgements}

\newpage

\tableofcontents
\newpage

\listoffigures
\newpage

\section{Introduction}
\label{sec:introduction}

\subsection{Project Description}
Pilots need to be notified of hazards and events that may require them to make changes to their flight plan. Currently there is a system in place where a flight planner must manually sort through information given as ‘Notice to Airman’ or ‘NOTAM’, to find only notices which are relevant to their flight. The goal of the software produced by over the course of the project was to limit the incoming data to a flight planner to only relevant data by utilising a publish and subscribe model.

The customer for the project was Snowflake Software, a company focusing on open data exchange, and various products based on mapping data. Within sprint meetings the company was represented by Eddie Curtis, who expressed the feature requests and feedback representing Snowflake as a whole.

The vision as stated by Snowflake was, “To develop a proof of concept push event service for delivering digital “NOTAM” updates to pilots; to demonstrate the feasibility of using Publish/Subscribe to communicate relevant NOTAMs to flight planners”.

The subscriptions to this service had to be based on the relevancy of NOTAMs to the recipients. This relevancy should be based on simple factors such as departure and arrival airports, to more complex temporal queries, limiting NOTAMs events to ones relevant only during the course of the flight. The software was required to use open standards for the publish and subscribe architecture, for example: OASIS Web Services Notification or WSN. The software also had to handle an estimated 10,000 messages per day. These messages come from two sources: a baseline NOTAM which arrives every 28 days from the Federal NOTAM Service and updated delta files which arrive every 15 minutes. 

\subsection{Background Information}

\subsubsection{Publish/Subscribe Architecture}

\begin{figure}
\begin{center}
\def\svgwidth{\columnwidth}
\input{broker.pdf_tex}
\end{center}
\caption{Publish/Subscribe Architecture with Broker}
\label{fig:example_net}
\end{figure}

Publish-subscribe is a messaging architecture that allows decoupling between the \emph{producers} and the \emph{consumers} of \emph{messages}. Messages are categorised into \emph{topics}; producers publish messages on a given topic with no knowledge of which (if any) subscribers the topic has, and consumers subscribe to a given topic with no knowledge of which (if any) producers are publishing messages with that topic.

This architecture is generally implemented using a \emph{broker}, which exists as middleware between the producers and consumers. Producers send messages on a given topic to the broker, which then decides which consumers the messages should be forwarded to. Consumers subscribe via the broker to topics they’re interested in, and the broker forwards relevant messages to them. This architecture has a number of benefits, including providing loose coupling between producers and consumers (since they need no knowledge of each other to make use of this architecture) and facilitating scalability by supporting arbitrary numbers of producers and consumers for any given topic.

In Java, brokers typically implement the Java Message Service (JMS) API provided by Java Enterprise Edition, which offers a consistent interface for communicating with many different message broker implementations.

\newpage

\section{Planning and Organisation}
\label{sec:planning}

The project was built using the Agile software development practices of Scrum, and each sprint followed a similar structure. First, the customer came up with ideas for features which could be worked on in the following sprint. Next, user stories based on these features were created, dividing up the requirements of each feature and helping to create acceptance criteria which would come together to prove acceptance that the feature was successfully implemented.

At the beginning of each sprint, a sprint meeting was held to both close the previous sprint and prepare for the following one. First, features implemented in the previous sprint were demonstrated to the customer, allowing them to track the progress of the project and decide whether the feature had been successfully implemented. In the event the customer was not satisfied with the work done in the sprint, the user stories were kept in the backlog for the following sprint. Otherwise, the customer discussed future features to be implemented and broke the feature up into user stories to be added to the backlog.

Various software was used to carry out this process, including JIRA (access to which was provided by the client) and Cucumber. JIRA is a tracker that manages user stories and their corresponding acceptance criteria as per each feature. User stories within JIRA are stored in the standardised “As a, I want, So that” format, with their acceptance criteria written as “Given, When, Then”. This format is also used by Cucumber for acceptance tests. Team members on JIRA can choose what user stories to complete or they can be assigned.

Cucumber was used to express the acceptance criteria for each story, allowing them to be converted into automated acceptance tests which could be run using JUnit. The Cucumber features are written in a language called Gherkin. Each Gherkin feature has a number of scenarios that test the aspects of that feature. The scenarios are written such that the implementation of the feature can change without having to rewrite the file. In the JUnit test file methods are called by Cucumber based off regex tags on the methods. For each story, when all acceptance tests passed within JUnit, the user story is removed from the backlog.

\newpage

\section{Conclusions}
\label{sec:conclusions}

\newpage

\section{Future Work}

\newpage

\addcontentsline{toc}{section}{References}
\sloppy
\printbibliography

\end{document}
